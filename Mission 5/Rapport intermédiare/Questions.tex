\documentclass[11pt]{article}
\usepackage[utf8]{inputenc}
\usepackage[T1]{fontenc}
\usepackage[french]{babel}
\usepackage{amsmath}
\usepackage[bookmarks={true},bookmarksopen={true}]{hyperref}
\usepackage{graphicx}
\usepackage[a4paper]{geometry}
\usepackage{listings}
\usepackage{amssymb}
\usepackage{amsmath,amsfonts}
	\lstset{frame=tb,
		language=Java,
 		aboveskip=3mm,
  		belowskip=3mm,
  		showstringspaces=false,
  		columns=flexible,
  		basicstyle={\small\ttfamily},
  		numbers=none,
 		numberstyle=\tiny\color{gray},
  		keywordstyle=\color{blue},
  		commentstyle=\color{dkgreen},
  		stringstyle=\color{mauve},
  		breaklines=true,
  		breakatwhitespace=true
  		tabsize=3
	}
\pagestyle{plain}
\setlength{\parindent}{5mm}
\usepackage{amsmath}
\usepackage{color}
\definecolor{dkgreen}{rgb}{0,0.6,0}
\definecolor{gray}{rgb}{0.5,0.5,0.5}
\definecolor{mauve}{rgb}{0.58,0,0.82}



\title{\textbf{Projet LSINF1121 -  Algorithmique et structures de données\\ - \\ Rapport intermédiaire Mission 5} \\ {\large Groupe 26}}
\author{Laurian \bsc{Detiffe} \\(6380-12-00)\and Sundeep \bsc{Dhillon} \\(6401-11-00)\and Alexis \bsc{Macq} \\ (5910-12-00) \and Xavier \bsc{Pérignon} \\ (8025-11-00)\and Thibaut \bsc{Piquard}\\(4634-13-00)\and Thomas \bsc{Wyckmans} \\ (3601-12-00)}
\date{date}
\date{\vspace*{25mm}
\includegraphics[scale=0.75]{logo.jpg}\\
		\vspace*{30mm}
		\begin{center}
		Année académique 2015-2016 \\	
		\end{center}}

\begin{document}
\thispagestyle{empty}

\maketitle
\thispagestyle{empty}
%\tableofcontents
%\setcounter{tocdepth}{3}
%\setcounter{page}{1}
%\newpage

\section*{Questions et réponses}
\begin{enumerate}

\item
\item
\item
\item
\item
\item 
\item
\item
\item

\item Quelles sont les différentes étapes d’un algorithme de compression de texte qui
prend en entrée un texte et fournit en sortie une version comprimée de ce texte à
l’aide d’un codage de Huffman ? Soyez précis dans votre description en isolant
chaque étape du problème. Précisez notamment pour chaque étape les structures
de données utiles et la complexité temporelle des opérations menées.\\\\

\item

\item \textbf{[Question liée spécifiquement au problème posé]} En quoi les deux classes qui
vous sont fournies, \texttt{\textbf{InputBitStream}} et \texttt{\textbf{OutputBitStream}}, peuvent-elles
être utiles pour le problème de compression et de décompression avec un codage
de Huffman ? La postcondition de la méthode \texttt{\textit{close}} dans la classe \texttt{\textbf{OutputBitStream}}
précise notamment que \textit{si le nombre de bits déjà écrits ne correspond pas à un
multiple de 8 (un octet), des bits à 0 sont écrits pour compléter l’octet courant}.
Quand la situation décrite peut-elle se présenter ? Quelle est la conséquence de
cette postcondition sur votre programme de compression de texte ? Quelle est la
conséquence de cette postcondition sur votre programme de décompression ?\\\\

\end{enumerate}
\end{document}